\clearpage\secrel{lol.VM in \py\ (LLVM backend)}\label{llvmpy}

For taking the first impression of pure managed compilation let's play with
\py\ only and LLVM compiler framework. The full source code you can found in
\href{https://github.com/ponyatov/o/blob/master/lol.py}{lol.py}.

\begin{lstlisting}
$ sudo apt install llvm
$ sudo pip install llvmlite
\end{lstlisting}

\noindent
If you are not friendly with bare LLVM, you can use online translator 
\url{http://ellcc.org/demo/index.cgi}\ lets you convert C/\cpp\ code into
LLVM assembly. And also you will need some entry tutorial on LLVM class model
itself you can find in \cite{llvmcore}. For simplicity, we'll use
\verb|llvmlite| binding to \py, and use some metaprogramming.

\pg
First, we can create only empty LLVM module:
\begin{lstlisting}[language=Python]
import llvmlite.ir as ir
# lol: Virtual FORTH Machine (LLVM portable)
module = ir.Module('lol')
print module
\end{lstlisting}
\begin{lstlisting}
; ModuleID = "lol"
target triple = "unknown-unknown-unknown"
target datalayout = ""
\end{lstlisting}
