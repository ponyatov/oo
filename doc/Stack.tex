\secrel{Stack (LIFO)}

The \term{stack} is data container mostly used in our VM itself: we do all data
exchange between system elements via global \term{data stack}.

\bigskip\noindent
Stack is data structure with LIFO data flow: \emph{Last In First Out}.\\
It must support two principle messages (methods):
\begin{description}[nosep]
\item[push ( -- o )]\ \\adds element to the
top of stack (to the end of \verb|.nest[]|)\\
in VM method returns stack itself to do sequential operations
\item[pop ( o -- )]\ \\return \emph{and delete} topmost
element (last pushed)
\end{description}

\clearpage\noindent
There are other optional methods we will implement:
\begin{description}[nosep]
\item[top ( o -- ) ]\ returns topmost element but leave it on stack
\item[flush ( \ldots o -- )]\ clean up data
\item[dup ( o -- o o )]\ duplicate
\item[swap ( o1 o2 -- o2 o1 )]\ swap two top elements 
\end{description}

\medskip\noindent
Don't forget that we \emph{delegate all basic manipulations} on data to base
Object class:

\begin{lstlisting}[language=Python]
class Container(Object): pass
class Stack(Container): pass
\end{lstlisting}
\begin{lstlisting}[language=Python]
class Object:
	def __init__(self,V): ... self.flush()
\end{lstlisting}
\begin{lstlisting}[language=Python]
class Object:
	def flush(self):
		# store attributes in form of key/value
		self.attr = {}	# clean
		# store nested elements (ordered) / stack
		self.nest = []	# clean
		# return itself for sequential operations
		return self
\end{lstlisting}
\begin{lstlisting}[language=Python]
def test_Stack_flush(): assert \
    Stack('flush test').flush().nest == []
\end{lstlisting}

\clearpage
\begin{lstlisting}[language=Python]
class Object:
	def push(self,object): self.nest.append(object) ; return self
\end{lstlisting}
