\clearpage\secrel{Compiler}

In \F\ term \term{compile} means something different than mainstream languages:
a user defines new words using existing ones. In classical \F\
implementation dynamic memory allocation model is extra simple: you can only
\emph{compile bytes to the end of vocabulary}, moving HERE pointer. We will use
\textit{executable vectors}\note{and attribute trees as a native representation
for programs written in mainstream languages} for storing user definitions, so
\emph{compiling will be done into the end of vector} held in \verb|COMPILE|
variable.

\bigskip
Don't confuse with \term{dynamic compilation} introduced later \ref{dynajit}.

\clearpage\noindent
The simplest compilation sample can be executable vector analogous to SmallTalk
block or argumentless lambda function:
\begin{lstlisting}[language=Forth]
[ 1 2 3 ] ??
\end{lstlisting}
\begin{lstlisting}
<stack:DATA>
	<vector:>
		<integer:1>
		<integer:2>
		<integer:3>
\end{lstlisting}
		