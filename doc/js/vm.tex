\clearpage\secrel{Implementing object VM in \js}\label{jsvm}\secdown

Here we will start from scratch. Just create empty \verb|~/VM.html|, open it it
Google Chrome browser, and press \keys{F12} to get \js\ \emph{console}:
\begin{lstlisting}[language=html]
<title>oFORTH/js</title>
<script>
// put your code here, restart by just page reload
</script>
\end{lstlisting}

\noindent
Just create stacks in one line:

\begin{lstlisting}[language=c++]
D = [ ]												// data stack
R = [ ]												// return stack
\end{lstlisting}

\clearpage
\begin{lstlisting}[language=c++]
W = { }												// vocabulary
\end{lstlisting}

\noindent
Let's add few VM commands as primitive functions:

\begin{lstlisting}[language=c++]
W.nop = function nop() {}			// empty VM command
W.bye = function bye() {}			// stop system
\end{lstlisting}

\begin{lstlisting}[language=c++]
W.words = function words() {	// just print vocab
	console.log(W); }

window.onload = function entry () {	// system init
W['words'](); console.log(W.nop); }
\end{lstlisting}

\begin{lstlisting}[language=c++]
{nop: f, bye: f, words: f}		// vocabulary
f nop() {}										// VM function
\end{lstlisting}

\pg Visual elements makes out interface must be embedded into .html 

\begin{lstlisting}[language=html]
<html>
	<head>
		<title>oFORTH/js</title>
		<style> </style>
		<script> // VM code here </script>
	</head>
	<body>
		<pre      id="log"> </pre>
		<textarea id="pad"> </textarea>
	</body>
</html>
\end{lstlisting}

\pg Visible window split into two areas:
\begin{description}[nosep]
\item[log] here we will print log and program messages.
\item[pad] input commands, five lines will be enough to edit command codes, but
will not be too big on a mobile phone screen.
\end{description}
\begin{lstlisting}[language=html]
<pre id="log">
oFORTH/js live web object system
(c) Dmitry Ponyatov &lt;dponyatov@gmail.com&gt;
github: https://github.com/ponyatov/o
</pre>
\end{lstlisting}
\begin{lstlisting}[language=html]
<textarea id="pad" rows=5
	placeholder="# enter your commands here">
</textarea>
\end{lstlisting}

\pg Console emulation will be good only with monospace font, so add this to CSS:

\begin{lstlisting}[language=html]
* { font-family:"Lucida Console", monospace;
    font-size:3vmin; }
\end{lstlisting}

\noindent
Web page for full-screen console should be tuned:
\begin{lstlisting}[language=html]
html,body {
	/* full-screen */
	margin:0; padding:0; height:100%; width:100%;
	/* disable scroll bars */
	overflow:hidden;
	/* black mono console */
	color:white; background-color:black; }
\end{lstlisting}

\pg PAD should be screen wide and larger font to be handy:
\begin{lstlisting}[language=html]
#pad { font-size:4vmin; width:100%;
       position:absolute; bottom:0; }
\end{lstlisting}

\noindent
On a mobile phone, the whole screen will be filled with PAD and screen keyboard,
and we can't use \keys{Enter} as it used for multiline command editing. So it
will be helpful to \emph{run anything in PAD every time you want just click on
LOG}. So LOG should have a visible border, as we yet have some problems with
HTML layout\note{LOG element covers only its content}:
\begin{lstlisting}[language=html]
#log { border: 0.5vh double white; height:80%; }
\end{lstlisting}

\pg To test click run, let's modify \verb|words()| to output on LOG not in
\verb|console.log|:
\begin{lstlisting}[language=html]
W.words = function words() {
	log.innerHTML += '\n';
	for (i in W) log.innerHTML += i+'\t';
	log.innerHTML += '\n'; }
	
var log; var pad;
window.onload = function entry () { // system init
	log = document.getElementById('log');
	log.onclick = W.words;	// register click event
	pad = document.getElementById('pad');
	W.words(); console.log(W.nop); }
\end{lstlisting}

\pg
\begin{lstlisting}[language=html]
html { clip:auto; position:absolute; }
#log { font-size:4vmin; height: 70%; width: auto; white-space: pre-wrap; }
#pad { font-size:5vmin; height: 30%; width:100%; }
\end{lstlisting}

\clearpage\secrel{PAD input parsing (classical approach)}\label{jsword}\secdown

Here we will expore classical \F\ approach used for user input parsing. As \F\
syntax is extra simple, we can easily write WORD in a few lines of code able to
read and parse PAD contents.

\begin{lstlisting}
W.word = function word() {
	log.innerHTML += '\n' + pad.value + '\n'; }

window.onload = function entry () {
	log.onclick = W.word;
\end{lstlisting}

\noindent Clicking on LOG area you will get full text entered in PAD. 

\secup


\secup
