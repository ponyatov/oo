\clearpage\secrel{Advantages of actor model}\secdown

\secrel{Fault tolerance}

\href{http://www.brianstorti.com/the-actor-model/}{Here} you can find a note on
Erlang language as a sample of actor model implementation. Erlang preaches <<let
it crash>> ideology of program development: some actor named \term{process} do
its job without tracking any failure cases can happen when a program executes.
If something goes wrong, the process simply crashes. But this situation will be
caught by another \term{supervisor} actor, which will do crashed process
healing, moving it to some consistent state like just restart the process into
an initial state.
 
\secrel{Scalable computation distribution}

\emph{Message passing is natively scalable} across distributed systems. As all
actors totally \emph{isolated} from each other, there is no matter on what
computing node every actor resides\note{with some overhead of transmitting
messages}. But don't be happy: you still have interprocess data dependency, and
connection link capacity still limits your ability to share your data between
distributed nodes.

\secup