\clearpage\secrel{Numbers}

Numbers in computers are a \emph{multiformat} beast. In every simple program
written in any language you can find multiple types of numbers. First of all,
numbers can be \term{integer}s, \term{float}ing point, and in \cpp\ small
numbers can be a \term{char}acter codes. Integers can be defined as
\term{dec}imal literals, and \term{hex} numbers or \term{binary string} if you
write programs for microcontrollers or drivers. In science computations and
cryptography, you can face with \term{arbitrary precision arithmetic}, where
numbers can have a number of signs fill the whole your hard drive.

For the first time we will focus on computer program translation and
transformation, so we will use only number types, available at CPU hardware
level and in low-level languages: short integers and limited precision floating
point. But some everyday use application will require fixed-point numbers (money
computations in accounting in the first place), and maybe special sorts of
numbers for digital signal processing.

\begin{lstlisting}[language=Python]
class Number(Primitive):
	def __init__(self,V):
		Primitive.__init__(self, V)
		self.value = float(V)				# use python float
\end{lstlisting}

\noindent
Here you can see use of floating point number, provided by implementation
language\ --- \py\ in our case.

\clearpage
\begin{lstlisting}[language=Python]
def test_Number_point(): assert \
	# check .value type
	type(Number('-0123.45').value) \
		== type(-123.45) and \			
	# compare numbers via abs(delta) < epsilon
	abs( Number('-0123.45').value - (-123.45) ) \
		< 1e-6

def test_Number_exp(): assert \
	# the same for '-01.23e+45' and '-01.23E+45' 
\end{lstlisting}

\clearpage
\begin{lstlisting}[language=Python]
class Integer(Number):
    def __init__(self,V):
        Number.__init__(self, '0')
        self.value = int(V) # use python integer
\end{lstlisting}
\begin{lstlisting}[language=Python]
def test_Integer(): assert \
    type(Integer('-012345').value) == type(-12345) and \
    Integer('-012345').value == -12345
\end{lstlisting}
Here we use \verb|==|\ direct comparision as we have no precision loss effect
on integers.

\bigskip\noindent
If you want you can define special classes for unsigned hex numbers and binary strings if you work with low-level programming.
