\clearpage\secrel{Vocabulary}\label{vocab}

\begin{lstlisting}[language=Python]
W = Map('FORTH')			# global vocabulary register

def test_FVM_W(): assert W.head() == '<map:FORTH>'
\end{lstlisting}

\emph{Word search in vocabulary} is one of prime in \F\ mechanics. And going to
objects we will use the same hint to dispatch messages. \F\ don't use vocabulary
in the execution of compiled definition, as all cross-word links were resolved
in compilation stage, and Map implementation efficiency impacts only on the
speed of interpretation and compilation. In case of message dispatch in pure
dynamic object/actor model it is not so, so for real-world applications, we need
to do a large effort on making vocabulary search at a speed of light. It's like
to do a dynamic call via vocabulary.

\bigskip
Returning to a prototype implementation, we need to add some syntax sugar to Map
code: let's redefine \verb|operator<<| as we will use it a lot for injecting VM
commands implemented in \py\ into the vocabulary.

\begin{lstlisting}[language=Python]
class Map(Container):
	def __lshift__(self,F):				# operator<<
		try: self[F.value] = F			# push object
		except AttributeError:			# fallback for 
			self[F.__name__] = VM(F)	#    VM command
		return self									# return modified
\end{lstlisting}
\clearpage\noindent
Then we can test VM command injection into new empty vocabulary, using test
function as VM command (any function can be used as command, if it follows  
\verb|def command(<nothing>)| template):
\begin{lstlisting}[language=Python]
def test_Map_LL(): assert '%s' % \ # test tree out
	( Map('LL') << test_Map_LL ) == \
		'\n<map:LL>\n\ttest_Map_LL = <vm:test_Map_LL>'
\end{lstlisting}
