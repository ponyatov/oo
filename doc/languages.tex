\clearpage\secrel{Programming language selection}\secdown

\secrel{\py}\label{whyPython}

\py\ is high level and easy to use generic programming language.

\secrel{\js}

\js\ is widely used and has topmost 20/80 ratios for money earning in commercial
applications as the Web is a resin market. The most attractive feature of \js:
you should not do anything to install execution environment, as a browser
pre-installed on any smart device including mobiles. Here JS is primary meta
target.

\clearpage\secrel{\F}\label{whyFORTH}

\F\ is the easiest language you can imagine not in usage but \emph{in
implementation}. Source code is just single words delimited by spaces, there is
no syntax by fact.
% 
% \bigskip
Interpreter
\begin{enumerate}[nosep]
  \item reads \term{input stream} word by word,
  \item search readed \term{word} in \term{vocabulary}, and
  \item \term{execute}s it or \term{compile}s if \textit{compiler state} enabled
\end{enumerate}

\noindent
That's all! Some words has special \term{IMMED} flag and executes in any states.
This lets you switch off compiler mode or do some tricks with forward source
code lookup.

\bigskip\noindent
See more: first look in \cite{kelly,kellyru}: it's the best \F\ intro book 

\secdown
\secrel{Factor}

By the fact, we use assembly-like stack machine script to initialize the system,
and this language is more close to \href{http://factorcode.org/}{Factor}
language:
\url{http://factorcode.org/}

\bigskip\noindent
The design of Factor is close to your ideas: fully dynamic, object-oriented,
works on top of stack machine and have feel of \ST\ and Lisp
\term{homoiconicity}\note{\url{http://en.wikipedia.org/wiki/Homoiconicity}}.

\secup

\secup