\clearpage\secrel{Colon definitions}

\begin{lstlisting}[language=Forth]
\ simple colon definition with VM commands
: init ( -- ) nop bye ;
\end{lstlisting}

\noindent This \term{colon definition} creates new word with name placed after
colon. 

\begin{lstlisting}[language=Python]
def t_WORD(t):
    r'[a-zA-Z0-9_\?\.\[\]\:\;]+' # expand lexer
    return Symbol(t.value)
\end{lstlisting}

% \pg 

\begin{lstlisting}[language=Python]
def test_colon_def():
    D.flush()
    INTERPRET(''' : init ( -- ) nop bye ; ''')
\end{lstlisting}

\begin{lstlisting}[language=Python]
def colon():
	# fetch new word name
	WORD() ; WN = D.pop().value
	# non-empty COMPILE points to compilation mode
	global COMPILE ; W[WN] = COMPILE = Vector(WN)
W[':'] = Fn(colon)
\end{lstlisting}

\medskip\noindent
: word starts new definition with placing \emph{named} vector in vocabulary. As
current definition already pushed in vocabulary before definition ended, we can
use its name for compiling recursion calls.

We'll compile colon definitions into vectors, but not memory image like
classical \F\ systems does. Any callable object can be executed if it has
special \verb|__call__(self)| method.
