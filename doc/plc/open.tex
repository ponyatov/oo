\clearpage
\secrel{PLC: Industrial Automation}\secdown

\term{Programmable Logic Controller} (PLC) is a class of devices were created in
the late of 1960's targetted for replacing relay-based automatics in industrial
control. PLC widely used as it has standardized set of software tools,
programming languages and protocols shared between multiple manufacturers by IEC
1131-3 standard.

\begin{description}
\item[simulator \& tutorial]\ \\
\url{http://www.plcsimulator.net/}
\item[OpenPLC]\ \\
\url{www.openplcproject.com/getting-started-linux}
\end{description}

\pg With rising market for hobby and home automation, IoT infrastructure and
embedded Linux, we can still use established practices from industrial
automation mixing it with extra cheap hardware and DIY connected devices. The
most interesting hardware is FPGA-based PLC able to process signals and do
parallel control at amazing speeds comparing to common sequential software
implementations based on microcontrollers and SoCs with Von-Neumann computer
architecture.

\pagebreak\secrel{Installing OpenPLC on Linux}

\url{http://www.openplcproject.com/getting-started}

\begin{lstlisting}
~$ git clone https://github.com/thiagoralves/OpenPLC_v2.git
\end{lstlisting}

\secup
