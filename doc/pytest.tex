\clearpage\secrel{Unit testing for \py\ programs}\label{pytest}

As you can see in this manual we use \term{unit test}s for demonstrating VM
elements semantics written in \py. It can be strange and frustrating for
beginners. For example, it will be more simple to just print sample elements and
include what you should see, like this:
\begin{lstlisting}
>>> print Object('test')
<object:test>
\end{lstlisting}

\noindent
The key advance of using unit tests is: you always
\begin{itemize}[nosep]
  \item 
can get full demo samples of \emph{code usage in real cases}, and
  \item 
\emph{automatically check} if everything works well and especially \emph{not
broken after some changes}
\end{itemize}

\noindent
I use py.test package, available via
\begin{lstlisting}
$ sudo pip install pytest
$ py.test -v VM.py
\end{lstlisting}
If you use Eclipse, configure it via\\
\menu{Window>Preferences>PyDev>PyUnit}\\
\menu{Test runner>py.test runner}\\
\menu{Parameters for test runner>\{live it empty\}}

\medskip\noindent
and you can run any \py\ script via \keys{Ctrl+F11}\ with unit testing:\\
\menu{Run>Run As>Python unit test}
