\clearpage\secrel{Build from source (files)}

\begin{tabular}{l l}
Makefile & build scripts \\
.gitignore & files masks for generated and temp files (for git) \\
micro.c & Virtual Machine source code\\&extra portable, written in strict
ANSI C'89\\
h.h c.c & shared code \\
compiler.lex & compiler source code (flex/\cpp) \\
FORTH.uF & \F-system source code\\&written in \term{compiling script} syntax\\
\end{tabular}

\medskip\noindent
Install MinGW or GNU toolchain, and build it using
\begin{lstlisting}
$ cd ~/o/micro ; [mingw32-]make
\end{lstlisting}

\clearpage\secrel{Run system}
You will get few .exe\note{.exe required for Windows, and have no matter for
Linux} files in current directory:

\medskip\noindent
\begin{tabular}{l l}
uFORTH.exe & emulator to run on your host system \\
compiler.exe & simple bytecode compiler using \F-like script \\
FORTH.bc & demo system compiled into bytecode
\end{tabular}

\medskip\noindent
\verb|.bc| file is optional: resulting \verb|uFORTH.exe| was built as
full-functional standalone executable, and all required bytecode was
in-compiled. But \emph{you can run your own programs in .bc files} as:
\begin{lstlisting}
$ ./compiler.exe < YourProgram.uF > compiling.log
$ ./uFORTH.exe YourProgram.bc
\end{lstlisting}
