\secrel{base Object class}\label{Object}\secdown

The whole system built on top of \emph{universal base object}: 

\begin{lstlisting}[language=Python]
class Object:
  def __init__(self,V):
    self.type = self.__class__.__name__.lower()
    self.value = V
    self.attr = {} ; self.nest = []
  def __repr__(self):
    return '<%s:%s>' % (self.type,self.value)
\end{lstlisting}

\noindent
As you can see in \cite{budd}\ \emph{any object} must have two elements:
\begin{description}%[nosep]
\item[type]\ points to class of object\\
what '11' means? is it integer, number, or string of two `1`s\,?\\
we can't detect this directly, so we require some \term{type tag}

What type should this tag have itself? By design, it should be a pointer to some
Class object used for current object creation. But if we need only \emph{a fixed
set of types pre-built into the system}, we can downsize it just to tiny integer
value, indicates type \cite{STFPGA}. As we implement bootstrap VM
\ref{bootstrapVM} in \py, we can use a pointer to the class, provided by \py
VM. But now \emph{we are targetted on simplicity and not efficiency}, so we just
use dumb \verb|std::string| available even in \cpp.

\item[value]\ holds object value (in case of primitive object)\ \\
the \term{primitive} types like numbers and strings need only single variable to
do it's job

\item[attr]\ attr\verb|{}|ibutes (key/value \emph{map} data container)
\item[nest]\ nest\verb|[]|ed elements (\textit{ordered} list/\emph{vector} +
\emph{stack})

To write programs we need some \term{composite} data containers can hold
collections of data. First of all it is \F\ \term{vocabulary} holds all named
elements, like variables, elementary commands available from VM, and user
compiled functions.

To do it we can define special data classes like \verb|dict{}| and \verb|list[]|
available in \py. But as we are going to play with \term{program
transformation} \ref{PT}\ and dynamic OOP \ref{DynOOP}\ implementation, we'll
embed two primary collection elements \emph{directly into base class}.
 
\end{description}

\noindent
For the first view base class is over-bloated: base Object instance can work as
a single data element, map, vector, and stack simultaneously. But this
universality is the power. Our goal is pure dynamic scripting language with live
object database. This base class gives a lot of functionality from scratch,
including tagging for attribute grammar processing, and storing nested elements
for AST, tree and graph storage.

\begin{lstlisting}
>>> print Object('test')
<object:test>
\end{lstlisting}

\secup